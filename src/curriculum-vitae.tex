% Copyright 2019 Valentin Lahaye
%
% This work may be distributed and/or modified under the
% conditions of the LaTeX Project Public License, either version 1.3
% of this license or (at your option) any later version.
% The latest version of this license is in
%   http://www.latex-project.org/lppl.txt
% and version 1.3 or later is part of all distributions of LaTeX
% version 2005/12/01 or later.

\documentclass{tccv}
\usepackage[utf8]{inputenc}
\usepackage[french]{babel}
\usepackage[T1]{fontenc}
\usepackage{multicol}

\begin{document}

\part{Valentin Lahaye}{Data Scientist Junior}

\section{Formations}
\begin{yearlist}

\item[Génie logiciel]{2016 -- 2018}
     {Master Informatique}
     {Université de Bordeaux}

\item{2013 -- 2016}
     {Licence Informatique}
     {Université de Bordeaux}

\end{yearlist}

\section{Expériences}
\begin{eventlist}

\item{Novembre 2018 -- Décembre 2019}
     {Hadrian Advisors SA}
     {Data Scientist Junior}

Analyse et manipulation de séries temporelles de petite et moyenne taille. Réalisation en autonomie d'un modèle de régression permettant la prédiction du résultat des courses hippiques aux \'Etats-Unis. Utilisation et paramétrage de modèles à base d'arbres décisionnels. Emploi de techniques de sélection de features, de détection d'outliers et d'optimisation autonome d'hyper-paramètres. Mise en place et manipulation d'outils modernes de versionnage de code (\texttt{Git}) et de données (\texttt{DVC}). Mise en production de la solution. Le modèle détermine correctement le vainqueur d'une course dans 30~\% des cas (contre 13~\% aléatoirement). Utilisation avancée du langage Python et de bibliothèques de calcul matriciel (\texttt{Pandas}, \texttt{NumPy}), de machine learning (\texttt{Scikit-learn}, \texttt{XGBoost}, \texttt{LightGBM}) et de modélisation de graphes (\texttt{NetworkX}). Création d'un outil de description et de génération de features. Réalisation d'un outil de parsing de CV. Collecte et mise en miroir de données financières.

\item{Avril 2018 -- Septembre 2018}
     {Thales AVS France SAS}
     {Stagiaire en ingénierie}

Stage en entreprise sur le Campus Thales Bordeaux au sein du centre de compétences Digitalisation et Missions. Réalisation et intégration d'une nouvelle\\ chaîne de production permettant la génération du logiciel embarqué à bord des hélicoptères militaires Chinook Mark 6 de la RAF.

\end{eventlist}

\personal
    {92 rue Peydavant\newline Appartement 306\newline 33400 -- Talence}
    {07 86 62 93 96}
    {valentin.lahaye@gmail.com}
    {lahaye.pro}
    {git.io/vallahaye}
    {Permis B et véhicule personnel}

\section{Compétences}
\begin{factlist}

\item{Languages}{\texttt{Python • C • Java}}

\item{Bibliothèques}{\texttt{Pandas • NumPy}\newline\texttt{Scikit-learn • NetworkX}\newline\texttt{LightGBM • XGBoost}}

\item{Outils}{\texttt{Git • DVC • Jupyter}\newline\texttt{Docker • Vagrant}}

\item{Plateformes}{\texttt{Linux}}

\end{factlist}

\section{Langues}
\begin{factlist}

\item{Français}{Langue maternelle.}

\item{Anglais}{Pratique régulière, niveau B2.}

\end{factlist}

\section{Projets personnels}
\begin{yearlist}

\item{2019}
    {nomad \href{https://gitlab.com/emurub/nomad}{\scriptsize\faExternalLink}}
    {Projet de modification et d'amélioration du client de jeu
    \href{https://www.dofus.com/fr/plus-dofus/dofus-retro}{Dofus Retro}.}

\item{2019}
    {compendium \href{https://github.com/vallahaye/compendium}{\scriptsize\faExternalLink}}
    {Programme de téléchargement itératif de playlists hébergées sur YouTube.}

\item{2017}
    {libvigicrues \href{https://github.com/vallahaye/libvigicrues}{\scriptsize\faExternalLink}}
    {Bibliothèque permettant l'acquisition de données depuis le service
    \href{https://www.vigicrues.gouv.fr}{Vigicrues}.}

\end{yearlist}

\section{Loisirs}
\begin{factlist}

\item{Sport}{Pratique régulière de l'escalade.}

\item{Hackathons}{Participation aux événements \href{https://www.nuitdelinfo.com}{\bfseries Nuit de l'Info} de 2013 à 2017 ainsi qu'au hackathon \href{http://wdmh.fr}{\bfseries WDMH} 2017.}

\end{factlist}

\end{document}
